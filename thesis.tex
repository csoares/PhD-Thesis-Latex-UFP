 \documentclass[oneside,12pt]{Classes/UFP}

%% Defina as propriedades do PDF gerado
\ifpdf
    \pdfinfo { /Title  (UFP THESIS)
               /Creator (TeX)
               /Producer (pdfTeX)
               /Author (Christophe Soares csoares@ufp.edu.pt)
               /CreationDate (D:20120101000000)  %format D:YYYYMMDDhhmmss
               /ModDate (D:20121101120000)
               /Subject (Como escrever uma tese em LaTeX)
               /Keywords (PhD, Thesis)}
    \pdfcatalog { /PageMode (/UseOutlines)
                  /OpenAction (fitbh)  }
\fi

% Titulo

\title{Thesis Template in \LaTeXe for \\ [1ex] % espaçamento
	Universidade Fernando Pessoa
}

\ifpdf
  \author{\href{mailto:csoares@ufp.edu.pt}{Christophe Soares}}
  \collegeordept{\href{http://fct.ufp.pt/t}{Faculdade de Ciências e Tecnologia}}
  \university{\href{http://www.ufp.pt}{Universidade Fernando Pessoa}}
% insert below the file name that contains the crest in-place of 'UnivShield'
  \crest{\includegraphics[width=25mm]{UFP}}
\fi

\degree{Doctor of Philosophy / Master of Science}
\degreedate{Date to be defined}

% turn of those nasty overfull and underfull hboxes
\hbadness=10000
\hfuzz=50pt

% Put all the style files you want in the directory StyleFiles and usepackage like this:
\usepackage{StyleFiles/watermark}

% espaçamento de um e meio entre linhas
\onehalfspacing

\begin{document}

% defina a lingua do documento
%\language{english}


% Documento em fase de edição (comentar ou descomentar consoante precisar
\watermark{DRAFT COPY ONLY}


\maketitle

%set the number of sectioning levels that get number and appear in the contents
\setcounter{secnumdepth}{3}
\setcounter{tocdepth}{3}

\frontmatter % book mode only
\pagenumbering{roman}

% empty page
\newpage
\thispagestyle{empty}
\mbox{}

\include{Chapters/Abstract/abstract}
\include{Chapters/Dedication/dedication}
\include{Chapters/Acknowledgement/acknowledgement}

\tableofcontents
\listoffigures
\listoftables
\printnomenclature  %% Print the nomenclature
\addcontentsline{toc}{chapter}{Nomenclature}

\mainmatter % book mode only
\include{Chapters/Introduction/introduction}
% \pagebreak[4]
% \hspace*{1cm}
% \pagebreak[4]
% \hspace*{1cm}
% \pagebreak[4]

\chapter{My First Chapter But Note The Numbering ...}
\graphicspath{{Chapters/Chapter1/Chapter1Figs/PNG/}{Chapters/Chapter1/Chapter1Figs/PDF/}{Chapters/Chapter1/Chapter1Figs/}}

\section{First Paragraph}
And now I begin my first chapter here ...

Here is an equation\footnote{footnote test}:
\begin{eqnarray}
CIF: \hspace*{5mm}F_0^j(a) &=& \frac{1}{2\pi \iota} \oint_{\gamma} \frac{F_0^j(z)}{z - a} dz
\end{eqnarray}


\section{Second Paragraph}
and here I write more ...\cite{texbook}

\subsection{sub first paragraph}
... and some more ...

Now I would like to cite the following: \cite{latex} and \cite{texbook}
and \cite{Rud73}.

I would also like to include a picture ...

\begin{figure}[!htbp]
  \begin{center}
    \leavevmode
    \ifpdf
      \includegraphics[height=6in]{aflow}
    \fi
    \caption{Airfoil Picture}
    \label{FigAir}
  \end{center}
\end{figure}

So as we have now labelled it we can reference it, like so (\ref{FigAir}) and it
is on Page \pageref{FigAir}. And as we can see, it is a very nice picture and we
can talk about it all we want and when we are tired we can move on to the next
chapter ...

I would also like to add an extra bookmark in acroread like so ...
\ifpdf
  \pdfbookmark[2]{bookmark text is here}{And this is what I want bookmarked}
\fi


\chapter{My Second Chapter}

\graphicspath{{Chapters/Chapter2/Chapter2Figs/PNG/}{Chapters/Chapter2/Chapter2Figs/PDF/}{Chapters/Chapter2/Chapter2Figs/}}

\section{First Section}
\markboth{\MakeUppercase{\thechapter. My Second Chapter }}
And now I begin my second chapter here ...

\begin{table}[tbh!]
\caption{Table} 
\label{tab:demo-1}
\centering
\begin{tabular}{l*{6}{c}r}
\hline
Team              & P & W & D & L & F  & A & Pts \\
\hline
FC Porto & 6 & 4 & 0 & 2 & 10 & 5 & 12  \\
Celtic            & 6 & 3 & 0 & 3 &  8 & 9 &  9  \\
FC Copenhagen           & 6 & 2 & 1 & 3 &  7 & 8 &  7  \\
SL Benfica     & 6 & 2 & 1 & 3 &  5 & 8 &  7  \\
\end{tabular}
\end{table}

\section{Second Section}
\markboth{\MakeUppercase{\thechapter. My Second Chapter }}
And here I write more ...

\subsection{first subsection in the Second Section}
... and some more ...

\subsection{second subsection in the Second Section}
... and some more ...

\subsection{third subsection in the Second Section}
... and some more ...

\include{Chapters/Chapter3/chapter3}
\def\baselinestretch{1}
\chapter{My Conclusions ...}

\graphicspath{{Chapters/Conclusions/ConclusionsFigs/PNG/}{Chapters/Conclusions/ConclusionsFigs/PDF/}{Chapters/Conclusions/ConclusionsFigs/}}


\def\baselinestretch{1.66}

Here I put my conclusions ...


\backmatter % book mode only
\appendix
\chapter{Appdx A}

and here I put a bit of postamble ...


\chapter{Appdx B}

and here I put some more postamble ...



%escolha um dos 3 estilos de bibliografia
\bibliographystyle{plainnat}
%\bibliographystyle{Classes/UFPbib}
%\bibliographystyle{Classes/jmb} % bibliography style

\renewcommand{\bibname}{References} 	% personalizar o nome da secção das referências bibliográficas
\bibliography{References/references} 		% Caminho para o bibtex

\end{document}
